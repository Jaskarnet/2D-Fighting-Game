\chapter{Wstęp}
\section{Wprowadzenie}
Świat gier komputerowych, w szczególność świat gier on-line, budowany jest zwykle w oparciu o różnego rodzaju mechanizmy rywalizacji. Dzięki tym mechanizmom samo granie przestaje mieć formę beztroskiej rozrywki, a staje się czymś więcej -- sposobem na sprawdzenie własnych możliwości, testem osiągniętej sprawności, okazją do zbudowania pozycji itp. Od wprowadzonych zasad oraz zakresu interakcji występujących między graczami częstokroć zależy, czy grono użytkowników danej gry powiększać się będzie o kolejne grupy fanów, czy nie zdobędzie ona żadnej popularności. Fakt ten też nie może specjalnie dziwić, gdyż rywalizacja jest uważana za część ludzkiej egzystencji, zaś uczestnictwo w cyfrowych konfliktach to tylko realizacji głęboko zakorzenionej potrzeby.

Wydawaniu gier komputerowych niezmiennie towarzyszy tworzenie się wokół nich lokalnych społeczności, zrzeszających osoby o podobnych zainteresowaniach. Przykładem takiej społeczności jest, ciesząca się zasłużoną renomą, społeczność FGC (ang.~\emph{Fighting Game Community}. Jej członkowie, włącznie z autorem niniejszej pracy, rywalizują ze sobą, dzielą się wskazówkami, transmitują swoje rozgrywki oraz, co równie istotne, budują nowe relacje przyjacielskie. Dla wielu z nich udział w turniejach to nie tylko okazja do rywalizacji, ale także pretekst do spotkania ze starymi znajomymi dzielącymi tę samą pasję.

Pojawiający się w angielskiej nazwie grupy termin \emph{Fighting Games} wskazuje na pewien gatunek gier, wokół których skupia się ta społeczność. Termin ten tłumaczony jest na język polski jako ,,bijatyki''. Jest to gatunek gier komputerowych, w których rywalizacja pomiędzy graczami odbywa się na wzór rywalizacji w sztukach walki -- gracze wykonują ruchy awatarami, zadając sobie ciosy, blokując ataki i stosując uniki. Tego typu gry mogą toczyć się w wirtualnym świecie o różnej złożoności, od dwuwymiarowej sceny, poprzez rozbudowane mapy aż po trójwymiarowe środowiska renderowane z dokładnością bliską filmowanej rzeczywistości.

Wskazane powyżej fakty dotyczące kultury bijatyk, jak również budowany od dziecka sentymentem do gier stanowiły kluczowy powód do zdefiniowana oraz wyboru przez autora tematu niniejszej pracy dyplomowej. Temat ten zakłada zaprojektować gry z gatunku bijatyk, przy założeniu, że oferować ona będzie prostą, dwuwymiarową wizualizację sceny oraz proste sterowanie. Gra powinna umożliwiać prowadzenie rozgrywek online, w związku z czym konieczne będzie opracowanie jej interfejsu sieciowego. Przykładem tego rodzaju gry jest FOOTSIES: \url{https://store.steampowered.com/app/1344740/FOOTSIES_Rollback_Edition/}. 
% TO DO: można tutaj opisać tę grę nieco szerzej, dodajaś zrzuty ekranu oraz opisy "ciosów". Chodzi o to, by było wiadomo, jak wygląda bijatyka 2D.

Na jej przykładzie można zorientować się, o czym będzie niniejsza praca. Dokładniejszy opis jej celu i zakresu podano w kolejnym podrozdziale.

\section{Cel i zakres pracy}
Celem niniejszego projektu jest opracowanie oraz pełna implementacja dwuwymiarowej gry zręcznościowej, zaliczającej się do kategorii bijatyk. Gra ta ma umożliwiać rywalizację pomiędzy graczami w trybie online, przy wykorzystaniu uproszczonych rozwiązań w zakresie sterowania, grafiki i mechaniki walki.

Realizacja pracy odbywać się ma w następujących etapach:
\begin{itemize}
\item \textbf{Definicja Wymagań}
\begin{itemize}
	\item Przeprowadzenie szczegółowej analizy różnych możliwych dróg implementacji gry.
	\item Określenie wymagań funkcjonalnych i niefunkcjonalnych gry, uwzględniając rozgrywkę online, uproszczone sterowanie, grafikę i mechanikę walki.
	\item Sporządzenie dokładnej specyfikacji projektu na podstawie zebranych informacji.
\end{itemize}
\item \textbf{Dobór Technologii i Środowiska Pracy}
\begin{itemize}
	\item Wybór odpowiednich technologii programistycznych i narzędzi do stworzenia gry.
	\item Konfiguracja środowiska programistycznego, które umożliwi efektywną pracę nad projektem.
\end{itemize}
\item \textbf{Projekt i Implementacja Gry}
\begin{itemize}
	\item Opracowanie projektu gry, w tym projektu graficznego, interfejsu użytkownika i mechaniki rozgrywki.
	\item Implementacja gry, włączając w to tworzenie kodu źródłowego, grafik i dźwięków.
\end{itemize}
\item \textbf{Przeprowadzenie Testów i Przygotowanie Wdrożenia}
\begin{itemize}
	\item Testowanie gry w celu wykrycia i poprawienia błędów oraz optymalizacji działania.
	\item Przygotowanie gry do wdrożenia, włączając w to proces pakowania i przygotowania do dystrybucji w trybie online.
\end{itemize}
\item \textbf{Stworzenie Dokumentacji}
\begin{itemize}
	\item Opracowanie kompleksowej dokumentacji projektu, w tym instrukcji obsługi, opisującej mechanikę gry oraz konfigurację.
	\item Dokumentacja techniczna, która opisuje strukturę kodu źródłowego i technologie użyte w projekcie.
\end{itemize}
\end{itemize}

\section{Układ pracy}
W rozdziale pierwszym przedstawiono ....
W rozdziale drugim opisano ....
W kolejnym, trzecim rozdziale, przedstawiono ....
W rozdziale czwartym zwrócono uwagę na ....
Ostatni, piąty rozdział, przeznaczono na podsumowanie.
Pracy towarzyszy przykładowy wykaz literatury oraz przykładowe dwa dodatki. 

