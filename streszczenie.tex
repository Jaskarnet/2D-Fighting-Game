\pdfbookmark[0]{Streszczenie}{streszczenie.1}
%\phantomsection
%\addcontentsline{toc}{chapter}{Streszczenie}
%%% Poniższe zostało niewykorzystane (tj. zrezygnowano z utworzenia nienumerowanego rozdziału na abstrakt)
%%%\begingroup
%%%\setlength\beforechapskip{48pt} % z jakiegoś powodu była maleńka różnica w położeniu nagłówka rozdziału numerowanego i nienumerowanego
%%%\chapter*{\centering Abstrakt}
%%%\endgroup
%%%\label{sec:abstrakt}
%%%Lorem ipsum dolor sit amet eleifend et, congue arcu. Morbi tellus sit amet, massa. Vivamus est id risus. Sed sit amet, libero. Aenean ac ipsum. Mauris vel lectus. 
%%%
%%%Nam id nulla a adipiscing tortor, dictum ut, lobortis urna. Donec non dui. Cras tempus orci ipsum, molestie quis, lacinia varius nunc, rhoncus purus, consectetuer congue risus. 
%\mbox{}\vspace{2cm} % można przesunąć, w zależności od długości streszczenia
\begin{abstract}
W pracy dokładnie opisano proces powstawania tytułowej gry. W pierwszej części pracy skupiono się na analizie założeń projektowych, definiując główne funkcje aplikacji, jej interfejs użytkownika oraz unikalne cechy, które wyróżniają ją na tle innych gier tego gatunku. Następnie szczegółowo omówiono etap implementacji, w~tym wybór i wykorzystanie technologii Java \cite{Java} i frameworku libGDX \cite{LibGDX}. W ramach tego etapu rozważano różnorodne aspekty techniczne, od opracowania mechanik rozgrywki, systemu kolizji, animacji postaci, po zarządzanie danymi i~interakcje sieciowe. Kolejna część pracy poświęcona jest testowaniu gry, zarówno w aspekcie funkcjonalnym, jak i w zakresie stabilności i wydajności. Przeprowadzone testy miały na celu zapewnienie, że gra spełnia wszystkie postawione założenia. Ostatni rozdział pracy zawiera podsumowanie całego procesu tworzenia gry, wyciąganie wniosków i zaleceń na przyszłość. Praca ta stanowi źródło wiedzy na temat procesu projektowania, implementacji i testowania gier komputerowych, z naciskiem na gry z kategorii dwuwymiarowych bijatyk i opisuje wyzwania związane z tworzeniem gier online.
\end{abstract}
\mykeywords

{
\selectlanguage{english}
\begin{abstract}
The thesis provides details of the process of creating the title game. It begins with an analysis of the project's foundational concepts, defining the application's main functions, user interface, and unique features that set it apart from other games in its genre. The implementation phase, including the selection and application of Java \cite{Java} technology and the libGDX \cite{LibGDX} framework, is thoroughly discussed. This part of the thesis considers various technical aspects, from game mechanics and collision systems to character animation, data management, and network interactions. Subsequent sections are dedicated to game testing, both functional and in terms of stability and performance, ensuring that the game meets all set objectives. The final chapter summarizes the entire game development process, drawing conclusions and recommendations for future work. This thesis serves as a knowledge resource on the design, implementation, and testing process of computer games, emphasizing 2D fighting games and the challenges of creating online games.
\end{abstract}
\mykeywords
}
