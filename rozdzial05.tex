\chapter{Podsumowanie}
Niniejsza praca dyplomowa skupiała się na procesie projektowania, implementacji i testowania dwuwymiarowej gry bijatyki z funkcjonalnością rozgrywki online. W ramach pracy, autor dokonał przeglądu istotnych koncepcji i technik wykorzystywanych w projektowaniu gier, szczególnie tych z gatunku bijatyk. Kładąc szczególny nacisk na aspekty techniczne, takie jak system kolizji, animacja postaci i zarządzanie rozgrywką multiplayer, autor przedstawia proces od koncepcji po realizację. W części praktycznej, opisano rozwiązania wykorzystane do napisania kodu gry. Praca obejmuje również podejście do testowania, co jest kluczowe dla zapewnienia jakości i płynności rozgrywki. Projekt ten okazał się dla autora nie tylko możliwością zastosowania teoretycznej wiedzy w praktyce, ale również platformą do rozwijania umiejętności programistycznych i zdobywania doświadczenia w tworzeniu gier. Praca ta podkreśla, jak ważne w procesie edukacyjnym jest praktyczne doświadczenie, które pozwala na głębsze zrozumienie i opanowanie skomplikowanych aspektów tworzenia gier.

\section{Napotkane problemy}
W trakcie procesu testowania programu na różnych konfiguracjach komputerowych, autor projektu napotkał problem związany ze złym wyświetlaniem tekstur. Symptomem tego problemu było pojawianie się czarnego prostokąta w miejscu niektórych animacji. Po dokładnej analizie potencjalnych przyczyn, stwierdzono, że kwestia ta dotyczy głównie dłuższych animacji. Ustalono, że długość animacji ma bezpośredni wpływ na rozmiar spritesheetu zapisywanego w zasobach gry. Dotychczasowe założenie, że każdy wiersz spritesheetu zawiera jedynie pojedynczy sprite, okazało się niewystarczające przy animacjach wymagających znacznej liczby sprite'ów, co prowadziło do tworzenia nieproporcjonalnie długich obrazków. Problem ten okazał się szczególnie zauważalny w przypadku architektury niektórych komputerów oraz wersji OpenGL zainstalowanej na tych systemach. W odpowiedzi na wykryte problemy, autor zdecydował się na modyfikację wszystkich spritesheetów, ustalając, że każdy wiersz będzie zawierał 8 sprite'ów, przy czym każdy spritesheet składa się z 8 takich wierszy. Taki kwadratowy układ spritesheetów skutecznie rozwiązał problem z wyświetlaniem tekstur na dotychczas problematycznych konfiguracjach sprzętowych.

Początkowy zamiar autora zakładał implementację Rollback Netcode \cite{Rollback} w trybie multiplayer, jednak ze względu na ograniczenia czasowe, ten element nie został w pełni zrealizowany. Rollback Netcode miał na celu symulowanie ruchów gracza w sytuacjach braku aktualnych danych o jego działaniach i korygowanie stanu gry po otrzymaniu tych informacji. W obecnej wersji kodu widoczne są początki implementacji tej funkcjonalności, gdzie wszystkie komendy skierowane do postaci gracza są rejestrowane w strukturze danych typu 'circular buffer', z możliwością ich cofnięcia. W przyszłości można rozważyć rozbudowę tej funkcjonalności o bardziej zaawansowany model przewidywania ruchów gracza.

%Tu moze jakis opis problemow i dodanie cytowania rollbacknetcode