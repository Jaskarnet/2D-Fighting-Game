\chapter{Podsumowanie}
Niniejsza praca dyplomowa skupia się na procesie projektowania, implementacji i testowania dwuwymiarowej gry bijatyki z funkcjonalnością rozgrywki online. W ramach pracy, autor dokonał przeglądu istotnych koncepcji i technik wykorzystywanych w projektowaniu gier, szczególnie tych z gatunku bijatyk. Kładąc szczególny nacisk na aspekty techniczne, takie jak system kolizji, animacja postaci i zarządzanie rozgrywką multiplayer, autor przedstawia proces od koncepcji po realizację. W części praktycznej, opisano rozwiązania wykorzystane do napisania kodu gry. Praca obejmuje również podejście do testowania, co jest kluczowe dla zapewnienia jakości i płynności rozgrywki. Projekt ten okazał się dla autora nie tylko możliwością zastosowania teoretycznej wiedzy w praktyce, ale również platformą do rozwijania umiejętności programistycznych i zdobywania doświadczenia w tworzeniu gier. Praca ta podkreśla, jak ważne w procesie edukacyjnym jest praktyczne doświadczenie, które pozwala na głębsze zrozumienie i opanowanie skomplikowanych aspektów tworzenia gier.