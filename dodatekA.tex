\chapter{Instrukcja wdrożeniowa}
\section{Uruchomienie z pliku wykonywalnego}
Aby pobrać plik wykonywalny gry należy wejść na stronę repozytorium \url{https://github.com/Jaskarnet/2DFightingGame.git}, następnie przejść do sekcji \texttt{release}, gdzie można znaleźć plik \texttt{.rar}. Po jego pobraniu i rozpakowaniu wystarczy uruchomić znajdujący się w nim plik wykonywalny \texttt{SunsetShowdown.exe}.

\section{Skompilowanie kodu}
\noindent \textbf{Wymagania wstępne:}
Java Development Kit (JDK), Gradle, środowisko programistyczne -- zalecane jest użycie IntelliJ IDEA, git -- potrzebny do klonowania repozytorium.

\noindent \textbf{Klonowanie repozytorium:}
Otwórz terminal. Sklonuj repozytorium, używając polecenia: \texttt{git clone https://github.com/Jaskarnet/2DFightingGame.git}. Przejdź do sklonowanego katalogu.

\noindent \textbf{Konfiguracja projektu:}
Otwórz sklonowany projekt w środowisku programistycznym. Upewnij się, że struktura projektu została poprawnie zaimportowana i że wszystkie potrzebne zależności są zainstalowane. Możliwe, że środowisko samo wykryje i zainstaluje potrzebne zależności.

\noindent \textbf{Uruchamianie gry:}
Znajdź główną klasę aplikacji, która znajduje się w ścieżce \texttt{desktop/src/com/mygdx/game/DesktopLauncher.java}. Uruchom projekt, korzystając z przycisku \emph{Run} w~IDE. Jeżeli uruchomienie programu zakończy się błędem, prawdopodobnie należy ustawić katalog roboczy na folder \texttt{assets}. W tym celu należy nacisnąć przycisk \emph{More Actions} (trzy kropki) znajdujący się w górnej części graficznego interfejsu obok nazwy uruchamianej konfiguracji i z menu kontekstowego wybrać opcję \emph{Edit} z zakładki \emph{Configuration}. Następnie w otwartym okienku należy znaleźć opcję \emph{Working Directory} i zmienić jej wartość na ścieżkę do folderu \texttt{assets} znajdującego się pod głównym folderem projektu. Po tej konfiguracji program powinien się skompilować i uruchomić poprawnie.

\noindent \textbf{Uwagi końcowe:}
Dokładne kroki mogą się różnić w zależności od konkretnego środowiska programistycznego i konfiguracji systemu. Jeśli środowisko automatycznie nie pobrało zależności, należy je uwzględnić.